\documentclass[12pt, letterpaper]{article}
\usepackage{graphicx}
\usepackage[
backend=biber,
style=alphabetic,
sorting=ynt
]{biblatex}
\addbibresource{bibliografie.bib}

\title{Demonstrații zero-knowledge de apartenență la mulțimi}
\author{Andrei Pârjol}
\date{April 2024}

\begin{document}

\maketitle

\section{Introducere}

\subsection{Obiective} Obiectivul lucrării de față este acela de a prezenta și studia conceptele și implementările curente pentru protocoalele zk-SNARK folosite în demonstrațiile de apartenență la mulțimi. Deși pot părea abstracte la prima vedere , această ramură de demonstrații (in eng. \emph{zero-knowledge proof of membership}) are o gamă largă de aplicații precum : anonimizarea tranzacțiilor cu criptomonede (e.g. protocolul Zcash pentru Bitcoin și protocolul Tornado Cash pentru Ether), votul electronic descentralizat și anonim (e.g. putem să demonstrăm ca avem dreptul să votăm fără să dezvăluim date personale) sau mai general, folosirea anonimă a unor servicii online (e.g. fără să folosim username/password).

\subsection{Contribuția personală}

Pentru a arăta relevanța ideilor prezentate în această lucrare am scris o librărie JavaScript care implementeaza arborii hash Merkle și procedurile de generare și verificare a demonstrațiilor pentru apartenență folosind SNARK-uri. De asemenea se propun și îmbunătățiri , folosind arbori hash "\emph{indexați}" care reduc adancimea arborelui și implicit numărul de apeluri la funcția hash folosită in circuitul algebric.

\pagebreak

\section{Fundamente teoretice}
\subsection{Scurt istoric}
\vspace{5mm}
    Termenul de zero knowledge a fost propus prima dată la mijlocul anilor 1980 de către cercetătorii  Shafi Goldwasser , Silvio Micali și Charles Rackoff de la Institutul de tehnologie din Massachusetts . Ei încercau sa rezolve problemele legate de sistemele de demonstrare interactive , sisteme teoretice în care o parte numită Prover încearcă să convingă o altă parte numită Verifier că o propoziție matematică este adevărată.

    Acest tip de sistem este numit interactiv deoarece cele două părți interschimbă mesaje în timpul procesului de demonstrare și la vremea respectivă o mare parte din muncă era îndreptată înspre asigurarea validității sistemului, adică rezolvarea cazului în care Prover-ul avea intenții malițioase și încearca să păcălească Verifier-ul în a crede o propoziție falsă.

	În sistemele de demonstrare interactive este presupus că Demonstratorul are putere de calcul nelimitată (informal toate problemele sunt fezabile) însă nu este de încredere și Verificatorul are putere de calcul limitată și este onest. Ce au făcut cei trei cercetători a fost să ia în considerare și cazul în care Verificatorul nu este de încredere și s-au întrebat ce informații poate să obțină Verificatorul după o demonstrație. O astfel de scurgere de informații este destul de gravă deoarece din ipoteză folosind aceste sisteme Verificatorul are acces la informații pe care în mod normal nu ar fi putut să le calculeze. 

	A fost fost propusă astfel implementarea unui nou sistem , zero knowledge , în care se demonstrează cunoașterea unei soluții la o problemă în loc de soluție în sine . După terminarea demonstrației Verificatorul nu învață nimic nou în afara faptului că Demonstratorul cunoaște soluția.\cite{greenandblazewebsite}

 \subsection{Definiție formală}
 
 Dat fiind un sistem de demonstrație (P,V) și un Limbaj L (astfel încât $x\in L$ să fie echivalent cu x este adevărat ), acest sistem este zero knowledge dacă satisface următoarele trei proprietăți: 


\textbf{Completitudine} : $x\in L$  Pr[V acceptă ] = 1 . x este acceptat cu probabilitate 1 atunci când avem un demonstrator și verificator onest  .


\textbf{Corectitudine} : $x\in L$  Pr[V acceptă ] = 1/n , $n\in N$ . x este acceptat cu probabilitate redusă/mică  atunci când avem o demonstrație mincinosă și un verificator onest.


\textbf{Zero Knowledge} : Pentru orice verificator V exista o simulare S astfel încât orice rezultat final sau intermediar obținut de V se poate obține și de către S. Informal V nu poate să calculeze nimic din ce nu putea să calculeze înainte de verificarea demonstrației.

\printbibliography
\end{document}
